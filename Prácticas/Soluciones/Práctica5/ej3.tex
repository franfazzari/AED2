\section{Ejercicio 3: k más chicos}

\begin{algorithm}[H]
\caption{
    \textbf{BuscarMinimos}(\textbf{in} A: arreglo(nat), \textbf{in} k: nat) $\to$ \textbf{out} res: arreglo(nat)
}
\begin{algorithmic}[1]
    \State res $\gets$ CrearArreglo(k) \Comment{$O(k)$}
    \State max $\gets$ BuscarMax(A) \Comment{$O(n)$}
    \For{i $\gets$ 1 \textbf{to} k} \Comment{$O(k)$}
        \State res[i] $\gets$ max
    \EndFor
    \For{i $\gets$ 1 \textbf{to} k} \Comment{$O(k)$}
        \For{j $\gets$ 1 \textbf{to} n} \Comment{$O(n)$}
            \If{A[j] $<$ res[i] $\land$ (i = 1 $\oLuego$ A[j] $>$ res[i-1])}
                \State res[i] $\gets$ A[j]
            \EndIf
        \EndFor
    \EndFor
\end{algorithmic}
\Complexity{$O(kn)$}
\end{algorithm}

Cuando $k > log(n)$ conviene primero ordenar el arreglo, por ejemplo con MergeSort que tiene complejidad $O(n log(n))$, y luego simplemente copiar en res los primeros $k$ elementos que tiene complejidad $O(k)$.

\section{Ejercicio 9: alumnos}

\subsection{}

\begin{algorithm}[H]
\caption{
    \textbf{OrdenarPlanilla}(\textbf{in/out} P: arreglo(alumno))
}
\begin{algorithmic}[1]
    \State CountingSort(P) \Comment{$O(n)$, clave de ordenamiento: alumno.puntaje, orden creciente}
    \State CountingSort(P) \Comment{$O(n)$, clave de ordenamiento: alumno.genero, orden decreciente}
\end{algorithmic}
\Complexity{$O(n)$}
\end{algorithm}

\subsection{}

Igual que la solución anterior ya que si los géneros están acotados, podemos hacer CountingSort ordenando por alumno.genero en tiempo lineal $O(n)$ donde $n$ es la cantidad de alumnos en la planilla.

\subsection{}

El lower bound $\Omega(n log(n))$ solo aplica a algoritmos de ordenamiento basadas en comparaciones. No estamos contradiciendo ese lower bound porque el algoritmo de ordenamiento utilizado, CountingSort, no se basa en comparaciones.
